
\section{Research Experience}

\outerlist{

\entrybig
	{\textbf{Galaxy Distribution around redMaPPer Clusters in SDSS}}{Shanghai, China}
	{Advisor: Ying Zu, Shanghai Jiao Tong University}{Jan 2020 -- Now}
\innerlist{
	\entry{Calculated cluster-galaxy cross-correlation using SDSS DR8 photometric catalog and redMaPPer cluster catalog, reproduced previous measurements of splashback radius and halo assembly bias and contributed to a paper submitted to MNRAS (arXiv:2012.08629).}
	\entry{Introduced isolation criteria to identify clusters suffering from projection effects, tested different ways of conducting isolation and confirmed our criteria could reduce the discrepancy between splashback measurements using redMaPPer clusters and simulations.}
	\entry{Currently working on building a simplified redMaPPer cluster finding algorithm and using mock data to test our isolation criteria.}
}

\entrybig
	{\textbf{Unified Modelling of the Galaxies and Hot Diffuse
	Gas in Cosmic Environments}}{Victoria, Canada}
	{Advisor: Arif Babul, University of Victoria}{July 2019 -- Now}
\innerlist{
	\entry{Used PyAtomDB to calculate the X-ray properties of intra-group medium in hydro simulations, including luminosities, temperatures, entropies, etc. Wrapped the codes into a python package \href{https://xigrm.readthedocs.io/}{XIGrM} and wrote detailed documentations for public usage (project website: \url{https://xigrm.readthedocs.io/}).}
	\entry{Analyzed a series of simulations with different stellar feedback models to see their influences on intra-group medium and their consistency with observations. Generated X-ray images and radial profiles of groups to investigate the morphological differences in detail.}
	\entry{Applied similar analysis to another cluster zoom-in simulation, found the cluster had abnormally high luminosity and pinned the problem on a group of hot but dense gases which was unusual.}
	\entry{Currently working on studying the origin and evolution history of those unusual gases.}
}

\entrybig
	{\textbf{Anisotropic Ejecta Distribution of Kilonova AT 2017gfo}}{Nanjing, China}
	{Advisors: Zi-Gao Dai and Bin-Bin Zhang, Nanjing University}{Oct 2017 -- Now}
\innerlist{
	\entry{Considered ISM extinction and processed the multi-band data collected by \texttt{Open Kilonova Catalog} to make the observables directly comparable with simulation results.}
	\entry{Built a semi-analytical model and used MCMC to determine the best fitting kilonova ejecta distribution in AT 2017gfo event. Our results showed previous spherical multi-components model was incomplete and helped to understand the behaviour of dynamical and wind ejecta during a binary neutron star merger.}
	\entry{Took relativistic Doppler effects into consideration when calculating observables, which was later proved to play an important role in shaping the observed light curve.}
	\entry{Currently working on using simulation which includes detailed radiation transport to validate our simplified model according to reviewer's advice.}
}

}
